Distributed systems has been adopted for building modern Internet services and cloud computing infrastructures, in order to obtain services with high performance, scalability, and reliability. Cloud computing SLAs require low time to identify, diagnose and solve problems in a cloud computing production infrastructure, in order to avoid negative impacts into the quality of service provided for its clients. Thus, the detection of error causes, diagnose and reproduction of errors are challenges that motivate efforts to the development of less intrusive mechanisms for monitoring and debugging distributed applications at runtime. 

Network traffic analysis is one option to the distributed systems measurement, although there are limitations on capacity to process large amounts of network traffic in short time, and on scalability to process network traffic where there is variation of resource demand.

The goal of this dissertation is to analyse the processing capacity problem for measuring distributed systems through network traffic analysis, in order to evaluate the performance of distributed systems at a data center, using commodity hardware and cloud computing services, in a minimally intrusive way. 

We propose a new approach based on MapReduce, for deep inspection of distributed application traffic, in order to evaluate the performance of distributed systems at runtime, using commodity hardware. In this dissertation we evaluated the effectiveness of MapReduce for a deep packet inspection algorithm, its processing capacity, completion time speedup, processing capacity scalability, and the behavior followed by MapReduce phases, when applied to deep packet inspection for extracting indicators of distributed applications.

\begin{keywords}
Distributed Application Measurement, Profiling, MapReduce, Network Traffic Analysis, Packet Level Analysis, Deep Packet Inspection
\end{keywords}
% Introdução: cenario, motivacao, problema, solucoes atuais, escopo do trabalho, lista de contribuicoes, organizacao do resto do artigo. 
% Chapter Structure
% 	Motivation
% 	Problems Statement
%	Contributions
% 	Dissertation Organization
\chapter{Introduction}
\label{ch:1_introduction}

\begin{quotation}[]{Lao Tzu}
The softest things in the world overcome the hardest things in the world.
\end{quotation}


\section{Motivation}
\label{sc:motivation}

Traditionally, cyber defense methods can be effective against ordinary and conventional types of attacks, yet may fail against innovative malicious techniques \cite{lakhina2005mining}. In order to be able to detect and avoid novel attacks and their variations, it is necessary to develop or improve techniques to achieve efficiency on resource consumption, processing capacity and response time. Moreover, it is crucial to obtain high detection accuracy and capacity to detect variations of malicious patterns. Recently, signal processing schemes have being applied to detect malicious traffic in computer networks \cite{Lu2009,Huang2009,Zonglin2009,david2011blind,da2012improved,tenorio2013greatest, vieira2017model}, showing advances in network traffic analysis.

Information security may consist of both technical and procedural aspects. The former includes equipment and security systems, while the latter corresponds to security rules and recommendations. Intrusion detection and intrusion prevention systems are security systems used, respectively, to detect (passively) and prevent (proactively) threats to computer systems and computer networks. Such systems can work in the following fashions: signature-based, anomaly-based or hybrid \cite{Huang2009,mudzingwa2012study}. Additionally, anomaly detection techniques can be categorized in classification, statistical, information theory and clustering based, according to \cite{bhuyan2014network,ahmed2016survey,osanaiye2016distributed}.

Cloud computing is a new rapidly evolving paradigm in the world of distributed networking and computation. The basic features of the cloud environment is providing elastic, on-demand and secure services for the end-users. While the first two requirements are rather well conceptualized and supported by the majority of the cloud platforms in use, security is a serious concern of the cloud providers and governmental organizations as well as academia and research community \cite{csa2016,higashi2015,gartner2015}. Although for the small and medium-sized enterprises (SME) the cloud environment is often the most cost-effective and easily scalable solution, the security and privacy of the sensitive data in cloud platforms are also not fully conceptualized, leading to obscure and incomplete security paradigms and solutions \cite{galibus2017offline}.

Additional security issues and requirements have to be considered when mobile clients are actively used in corporate cloud environment \cite{yovel2014}. Today organizations and enterprises are adopting the Bring-Your-Own-Device (BYOD) paradigm. The uncontrolled usage of the mobile devices represents a serious risk to the development of secure SME cloud platforms, being the bottleneck of the corporate information security system (ISS). While the enterprise cloud infrastructure based on the web interface can be protected by powerful third-party services, the corporate mobile client is usually light-weighted and generally less protected. 

The protection scheme used on a mobile device should be both computationally secure as well as resource-constrained due to battery power limitations \cite{khan2015cloud}. Therefore, encrypting files and generating keys on a mobile device is not considered a good solution. On the other hand, the protection schemes with good computational qualities lack the security analysis in many cases \cite{khan2014bss}. The common practice is the shadow user activities monitoring \cite{yovel2014}. However, the mobile device usage stays unprotected in all the proposed scenarios while in offline mode. When the mobile client goes offline with the sensitive corporate data on board all powerful cloud-based tools cannot help and the mobile client has to secure itself with its own limited resources. Moreover, due to the resources constraint, there is a crucial difference in strategy of online and offline mode protection.

The field of Mobile Money Transfer (MMT) is a growing market segment and has been motivating investments into security defense and fraud detection. Obtaining access to data sets of mobile transactions for research is a very hard task due to the intrinsic private nature of such transactions. Scientists and researchers must today spend time and effort in obtaining access to relevant data sets before they can research on such data set. However, PaySim \cite{lopez2016paysim} is a financial simulator that simulates mobile money transactions based on an original dataset and provides a synthetic dataset as an approach to such a problem.

Recent developments in science and technology have enabled the growth and availability of raw data to occur at an explosive rate. This has created an immense opportunity for knowledge discovery and data engineering research to play an essential role in a wide range of applications from daily civilian life to national security. Despite the actual high availability of information, the relevant information of some observations is generally of under much reduced dimensionality compared to available data sets. The extraction of relevant information by identifying the generating causes within classes of signals is useful for classification problems and for security analysis. 

The high availability of raw data increases the challenges related to big data analytics and to imbalanced data, which corresponds to data sets exhibiting significant imbalances of classes or rare events of some classes. The fundamental issue with the imbalanced learning problem is the ability of imbalanced data to significantly compromise the performance of most standard learning algorithms. Data analysis of imbalanced data is challenging for classification and prediction problems related to anomaly detection, novelty detection, fraud detection and attack detection. 

Considering that imbalanced data can be composed by rare or unknown events, adaptive techniques can be useful in order to improve models and be able to react to changes or unknown patterns and maintain acceptable performance for classification algorithms. Adaptive techniques are also important for online data analysis or streaming processing, where an incremental processing approach can be applied to deal with high throughput data analysis.

Compressed Sensing (CS) aims to obtain the most relevant information of a dataset, what makes it useful for compression, signal reconstruction, image processing and classification, such as in sparse representation classification (SRC) problems. Dictionary learning is related to CS but seeks to obtain compact and meaningful signal representations, known as dictionary, from learning algorithms or models. DL is a signal processing technique for sparse representation of signals as basis vectors, learning the representations from training data, as dictionaries. DL is based on the principle that some observations can be described by a sparse subset of atoms taken from a redundant dictionary, which represents the causes of some observations of the world. The sparse representation in terms of such dictionaries has attracted increased interest for compressive sensing and for solving problems such as denoising, compression, image processing, data decomposition, feature extraction and classification \cite{tosic2011dictionary, zhang2010discriminative, zhu2016coupled,ravishankar2011mr}. 

Dictionaries are either available analytically, or can be learned from a suitable training set. DL can be formulated as an adaptive algorithm that learns from training data that is updated to represent changes or to adapt to new patterns. Iterative dictionary learning has been used as a blind technique for feature extraction, improving classification without require feature selection or principal component analysis

In some applications, the data and its dictionary are multidimensional, e.g., when estimating jointly behavior of users in social networks. Computing tensor decompositions of multi-way datasets is particularly useful to extract hidden patterns and structure in multidimensional data analytics problems \cite{kolda2009tensor}. Tensor-based algorithms for dictionary learning can improve the performance for cases of multidimensional and separable data, regarding the dictionary identification rating, the required number of training samples and iterations for the optimization problem \cite{roemer2014tensor}. Existing dictionary learning schemes can be applied to multidimensional analysis and obtain valuable results. Additionally, the performance of tensor-based algorithms for recovering of a known separable dictionary outperform existing schemes when dealing with growing multidimensional datasets.


\section{Problem Statement}
\label{sc:problems}

Considering the above described landscape, this thesis outlines the development and evaluation of approaches based on signal processing for information security analysis, through methods to make the data discriminative and able to identify structures, hidden patterns and the most relevant information for network attack detection, mobile malicious behavior analysis and fraud detection in mobile money transactions.

In the context of anomaly-based schemes, this thesis proposes a statistical approach based on signal processing techniques for detection of malicious traffic in computer networks. Inspired by \cite{david2011blind,da2012improved}, this thesis models the network traffic using a signal processing formulation as a composition of three components: legitimate traffic, malicious traffic and noise, taking into account the incoming and outgoing traffic in certain types of network ports (TCP or UDP). The proposed technique is based on eigenvalue analysis, model order selection (MOS) and similarity analysis. In contrast to \cite{david2011blind,da2012improved,tenorio2013greatest}, MOS and eigenvalue analysis are applied to detect time frames under attack. In addition, we also evaluate the accuracy and performance of the proposed framework applied to a experimental scenario and to the DARPA 1998 dataset \citep{osanaiye2016distributed}, which is a well known network traffic dataset. Furthermore, this proposed approach has its accuracy evaluation based on eigen similarity analysis for extracting detailed information about accurate time and network ports under attack.

This thesis also proposes an architecture and approach based on user behavior analysis of mobile apps through Model Order Selection (MOS) \cite{tenorio2013greatest}, in order to detect possible threats, and to reduce the risks and the harm of the most common security threats to offline corporate mobile apps, which are the expired user misusing password and the intruder attack. Additionally, the behavioral analysis can indicate well known malicious behaviors, their variations, as well as novel attacks, which present low or high variance in comparison to legitimate user behaviors. The main target of this proposed solution is to provide a maximum defense at the minimal resource cost.

Considering that big data problems require techniques to deal with multidimensional data in order to make sense of structure and relationship of many dimensions, and also considering that a key challenge to use sparse coding and dictionary learning for classification is how to find proper dictionaries and coefficients that highlight discriminative structure and relationships of one dataset, we also propose a tensor-based dictionary learning method for fraud detection from imbalanced data, in order to apply the tensor decomposition for dictionary learning methods to evidentiate the discriminative sensing of a fraud detection dataset. Specifically, we propose to apply a sparse representation based classification (SRC) method through learning a tensor-based dictionary and evaluate the reconstruction error for fraud classification of mobile money transactions. We propose to use tensor-based dictionary learning for learning fraudulent and legitimate data separately, and apply the learned dictionaries to reconstruct a test signal and classify it as fraud or legitimate, according to the minimum reconstruction error measured by some metrics.


\section{Contributions}
\label{sc:contributions}

We analyze problems related to detection of information security issues and propose new approaches to improve malicious behavior detection through signal processing techniques. The results of the work presented in this thesis provide the following contributions:

\begin{enumerate}
	\item We propose an approach based on eigen similarity analysis for extracting detailed information about accurate time and network ports under network attack, and evaluated the accuracy and performance of the proposed framework applied to an experimental scenario and to the DARPA 1998 dataset;
	\item We discuss the computational complexity of the proposed framework and evaluate the required processing time for tested scenarios;
	\item We propose an architecture and techniques for offline behavioral analysis of a corporate mobile client security architecture;
	\item We discuss the processing time of the proposed framework for mobile devices;
	\item We propose a tensor-based dictionary learning approach for fraud detection in mobile payment transactions;
\end{enumerate}

\section{Thesis Organization}
\label{sc:organization}

This thesis is organized as follows. In Chapter \ref{ch:2_mos_eig_sim}, we propose a statistical approach based on signal processing techniques for detection of malicious traffic in computer networks, based on eigenvalue analysis, model order selection (MOS) and similarity analysis. Chapter \ref{ch:3_mobile} presents an evaluation of an approach and architecture based on user behavior analysis through Model Order Selection (MOS) \cite{tenorio2013greatest}, in order to detect possible threats in a mobile application. Chapter \ref{ch:4_tensor_dl} proposes a tensor-based dictionary learning approach for fraud detection from mobile payment transactions. Chapter \ref{ch:5_conclusionfuturework} draws the conclusions and the suggestions for future work. Furthermore, the Appendix \ref{apx:a_mos} presents mathematical concepts of examples of state-of-the-art MOS schemes and the Appendix \ref{apx:b_csf_fs} presents a critical factors analysis based on Principal Component Analysis (PCA) for visual discriminant analysis, and presents an approach based on Recursive Feature Elimination (RFE) combined with Support Vector Machine (SVM) \cite{hearst1998support}, applied to the survey that evaluates the IT governance of Brazilian public organizations, in order to identify the Critical Success Factors (CSF) for IT governance of the public sector according to TCU.
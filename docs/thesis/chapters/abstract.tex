The cost of all types of cyberattacks is increasing for global organizations. The Whitehouse of the U.S. government estimates that malicious cyber activity cost the U.S. economy between US\$57 billion and US\$109 billion in 2016. Recently, it is possible to observe an increasing in numbers of Denial of Service (DoS), botnets, malicious insider and ransomware attacks.

Accenture consulting argues that 89\% of survey respondents believe breakthrough technologies, like artificial intelligence, machine learning and user behavior analytics, are essential for securing their organizations. To face adversarial models, novel network attacks and counter measures of attackers to avoid detection, it is possible to adopt unsupervised or semi-supervised approaches for network anomaly detection, by means of behavioral analysis, where known anomalies are not necessaries for training models.

Signal processing schemes have been applied to detect malicious traffic in computer networks through unsupervised approaches, showing advances in network traffic analysis, in network attack detection, and in network intrusion detection systems. 

Anomalies can be hard to identify and separate from normal data due to the rare occurrences of anomalies in comparison to normal events. The imbalanced data can compromise the performance of most standard learning algorithms, creating bias or unfair weight to learn from the majority class and reducing detection capacity of anomalies that are characterized by the minority class. Therefore, anomaly detection algorithms have to be highly discriminating, robust to corruption and able to deal with the imbalanced data problem.

Some widely adopted algorithms for anomaly detection assume a Gaussian distributed data for legitimate observations, however this assumption may not be observed in network traffic, which is usually characterized by skewed and heavy-tailed distributions.

As a first important contribution, we propose the Eigensimilarity, which is an approach based on signal processing concepts applied to detection of malicious traffic in computer networks. We evaluate the accuracy and performance of the proposed framework applied to a simulated scenario and to the DARPA 1998 data set. The performed experiments show that synflood, fraggle and port scan attacks can be detected accurately by Eigensimilarity and with great detail, in an automatic and blind fashion, i.e. in an unsupervised approach.

Considering that the skewness improves anomaly detection in imbalanced and skewed data, such as network traffic, we propose the Moment-based Robust Principal Component Analysis (m-RPCA) for network attack detection. The m-RPCA is a framework based on distances between contaminated observations and moments computed from a robust subspace learned by Robust Principal Component Analysis (RPCA), in order to detect anomalies from skewed data and network traffic. We evaluate the accuracy of the m-RPCA for anomaly detection on simulated data sets, with skewed and heavy-tailed distributions, and for the CTU-13 data set. The Experimental evaluation compares our proposal to widely adopted algorithms for anomaly detection and shows that the distance between robust estimates and contaminated observations can improve the anomaly detection on skewed data and the network attack detection.

Moreover, we propose an architecture and approach to evaluate a proof of concept of Eigensimilarity for malicious behavior detection on mobile applications, in order to detect possible threats in offline corporate mobile client. We propose scenarios, features and approaches for threat analysis by means of Eigensimilarity, and evaluate the processing time required for Eigensimilarity execution in mobile devices.

\begin{keywords}
Anomaly Detection, Network Attack Detection, Imbalanced Data, Principal Component Analysis (PCA), Eigenvector Similarity, Robust Principal Component Analysis (RPCA).
\end{keywords}

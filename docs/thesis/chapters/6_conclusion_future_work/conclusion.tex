%%%%%%%%%%%%%%%%%%%%%%%%%%%%%%%%%%%%%%%%%%%%%%%%%%%%%%%%%%%%%%%%%%%%%%%%%%%%%%%%%%%%%%%%%%%%%%%%%%%%%%%%%%%%%%%%%%%%%%%%%%%%%%%%%%%%%%%%%
% na conclusao, evite apresentar os resultados mais fáceis de concluir. bons revisores olham o casamento do que vc escreve no abstract/intro com a conclusao. mantenha coerente. 
% em geral, resultados que sao aparentemente obvios, enfraquecem as outras contribuicoes. p.ex, "Os resultados mostraram que o MapReduce apresenta uma boa solução para utilizar computadores comuns para obter alta capacidade de processamento e lidar com a análise massiva de tráfego de rede." pode ser reformulado ou retirado da conclusao.
\chapter{Conclusion and Future Work}
\label{ch:conclusionfuturework}

\begin{quotation}[]{Lao Tzu}
The softest things in the world overcome the hardest things in the world.
\end{quotation}

Distributed systems has been adopted for building modern Internet services and cloud computing infrastructure. The detection of error causes, diagnose and reproduction of errors of distributed systems are challenges that motivate efforts to develop less intrusive mechanisms for monitoring and debugging distributed applications at runtime. 

Network traffic analysis is one option for distributed systems measurement, although there are limitations on capacity to process large amounts of network traffic in short time, and on scalability to process network traffic where there is variation of resource demand.

In this dissertation we proposed an approach to perform deep inspection in distributed applications network traffic, in order to evaluate distributed systems at a data center through network traffic analysis, using commodity hardware and cloud computing services, in a minimally intrusive way. Thus we developed an approach based on MapReduce, to evaluate the behavior of a JXTA-based distributed system through DPI.

We evaluated the effectiveness of MapReduce to implement a DPI algorithm and its completion time scalability to measure a JXTA-based application, using virtual machines of a cloud computing provider. Also, was deeply evaluated the performance of MapReduce for packet-level analysis and DPI, characterizing the behavior followed by MapReduce phases, its processing capacity scalability and speed-up, over variations of input size, block size and cluster size.

\section{Conclusion}
\label{sc:conc_conclusion}

With our proposed approach, it is possible to measure the network traffic behavior of distributed applications with intensive network traffic generation, through the offline evaluation of information from the production environment of a distributed system, making it possible to use the information from the evaluated indicators, to diagnose problems and analyse performance of distributed systems.

We showed that MapReduce programming model can express algorithms for DPI, as the Algorithm \ref{alg:JXTAperfmapper}, implemented to extract application indicators from the network traffic of a JXTA-based distributed application. We analysed the completion time scalability achieved for different number of nodes in a Hadoop cluster composed of virtual machines, with different size of network traffic used as input. We showed the processing capacity and the completion time scalability achieved, and also was showed the influence of the number of nodes and the data input size in the processing capacity for DPI using virtual machines of Amazon EC2, for a selected scenario.

We evaluated the performance of MapReduce for packet level analysis and DPI of applications traffic, using commodity hardware, and showed how data input size, block size and cluster size cause relevant impacts into MapReduce phases, job completion time, processing capacity scalability and in the speedup achieved in comparison against the same execution by a non distributed implementation.

The results showed that although MapReduce presents a good processing capacity using cloud services or commodity computers for dealing with massive application traffic analysis, but it is necessary to evaluate the behaviour of MapReduce to process specifics data type, in order to understand its relation with the available resources and the configuration of MapReduce parameters, and to obtain an optimal performance for specific environments.

We showed that MapReduce processing capacity scalability is not proportional to number of allocated nodes, and the relative processing capacity decreases with node addition. We showed that input size, block size and cluster size are important factors to be considered to achieve better job completion time and to explore MapReduce scalability, due to the observed variation in completion time provided by different block size adopted. Also, in some cases, the processing capacity does not scale with node addition into the cluster, what highlights the importance of allocating resources according with the workload and input data, in order to avoid wasting resources.

We verified that packet level analysis and DPI are Map-intensive jobs, due to Map phase consumes more than 70\% of the total job completion time, and shuffle phase is the second predominant phase. We also showed that using whole block as input for Map functions, it was achieved a poor completion time than the approach which splits the block into records.

\section{Contributions}
\label{sc:conc_contributions}

We attempt to analyse the processing capacity problem of measurement of distributed systems through network traffic analysis, the results of the work presented in this dissertation provide the contributions below:

\begin{enumerate}
	\item We proposed \textbf{an approach to implements DPI algorithms through MapReduce}, using whole blocks as input for Map functions. It was \textbf{shown the effectiveness of MapReduce for a DPI algorithm} to extract indicators from a distributed application traffic, also it was \textbf{shown the completion time scalability of MapReduce for DPI}, using virtual machines of a cloud provider;
	\item We developed \textit{JNetPCAP-JXTA} \citep{jnetpcapjxta}, an open source \textbf{parser to extract JXTA messages from network traffic traces};
	\item We developed \textit{Hadoop-Analyzer} \citep{Hadoopanalyzer}, an open source \textbf{tool to extract indicators from Hadoop logs and generate graphs} of specified metrics.
	\item We \textbf{characterized the behavior followed by MapReduce phases for packet level analysis and DPI}, showing that this kind of job is intense in Map phase and highlighting points that can be improved;
	\item We \textbf{described the processing capacity scalability of MapReduce for packet level analysis and DPI}, evaluating the \textbf{impact caused by variations in input size, cluster size and block size};
	\item We \textbf{showed the speed-up obtained with MapReduce for DPI}, with variations in input size, cluster size and block size;
	\item We \textbf{published five papers} reporting our results, as follows:
	\begin{enumerate}
		\item T. P. B. Vieira, D. F. Ten\'orio, J. P. C. L. da Costa, E. P. de Freitas, G. Del Galdo, and R. T. de Sousa J\'unior, \textit{Model Order Selection and Eigen Similarity based Framework for Detection and Identication of Network Attacks}. Journal of Networking and Computer Applications (JNCA), Vol 90, Jul 2017, Pages 26–41.
		\item T. Galibus, T. P. B. Vieira, E. P. de Freitas, R. Albuquerque, J. P. C. L. da Costa, R. T. de Sousa Jr, V. Krasnoproshin, A. Zaleski, H. Vissia, and G. del Galdo, \textit{Offline Mode for Corporate Mobile Client Security Architecture}. Mobile Networks and Applications, Springer, Mar 2017, Pages 1-17.
		\item K. H. C. Ramos, R. T. Sousa Jr, T. P. B. Vieira, and J. P. C. L. da Costa, \textit{DISCOVERING CRITICAL SUCCESS FACTORS FOR INFORMATION TECHNOLOGIES GOVERNANCE THROUGH BIBLIOMETRIC ANALYSIS OF RESEARCH PUBLICATIONS IN THIS DOMAIN}. Information (Yamaguchi), 2016
		\item K. H. C. Ramos, R. T. de Sousa Jr., T. P. B. Vieira, and J. P. C. L. da Costa, \textit{Identifying relevant IT governance critical success factors by means of bibliometric analysis}. 7th International Conference on Information, November 2015, Taipei, Taiwan.
		\item K. H. C. Ramos, T. P. B. Vieira, J. P. C. L. da Costa, and R. T. Sousa Jr., \textit{ANÁLISE MULTIDIMENSIONAL DE FATORES CRÍTICOS DE SUCESSO EM GOVERNANÇA DE TI NA ADMINISTRAÇÃO PÚBLICA FEDERAL À LUZ DOS DADOS DE CONTROLE EXTERNO}. in Proc. 12th International Conference on Management of Technology and Information Systems (CONTECSI), 2015, São Paulo, Brazil.
	\end{enumerate}
\end{enumerate}

\section{Future Work}
\label{sc:conc_futurework}

Because of time constraints imposed on the Ph.D. degree, this thesis addresses some problems, but some problems are still open and others are emerging from current results. Thus, the following issues should be investigated as future work:

\begin{itemize}
	\item ...;
	\item ....
\end{itemize}

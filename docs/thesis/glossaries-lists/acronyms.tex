% Example for Glossary:
% \newglossaryentry{latex}
% {
%     name=latex,
%     description={Is a mark up language specially suited 
%     for scientific documents}
% }
%
% To include acronyms into the list of symbols, do as follow:
% * add [parent=acronymslists] after every "\newacronym";
% * Comment the line "\listadeacronimos" in file "thesis.tex";
% * Remove the comment the following line:
%\newglossaryentry{acronymslists}{name={Acronyms list},description={\glspar},sort=c}


\newglossaryentry{maths}
{
    name=mathematic,
    description={Mathematics is what mathematicians do}
}

% Example for Acronyms:
% \newacronym{label}{Acronym}{Full name}
\newacronym{uti}{UTI}{Unidade de Tratamento Intensivo}
\newacronym{scuba}{SCUBA}{Self-Contained Underwater Breathing Apparatus}
\newacronym{radar}{RADAR}{RAdio Detection And Ranging}
\newacronym[description={Data Link layer/protocols}]{dl}{DL}{Data link}


% In the text:
%     use \gls{label}, for an first time extend name, plus acronyms
%     use \acrshort{label}, for an acronym only
%     use \acrlong{label}, for a full name only
% In the text , for plural:
%     \glspl{duck}
% In the text , for Caps and/or plural:
%     \Gls{duck}
%     \Glspl{duck}



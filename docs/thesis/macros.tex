% Figure commands
\newcommand{\figOneCenter}[4]{
	\begin{figure}[htb]
		\centering %
		\scriptsize %
		\includegraphics[#1]{#2}
		\caption{#3}
		#4
	\end{figure}
}

\newcommand{\figMultCenter}[2]{
	\begin{figure}[htb]
		\centering %
		\scriptsize %
		#1
		#2
	\end{figure}
}

\newcommand{\figOneCenterHere}[4]{
	\begin{figure}[H]
		\centering %
		\scriptsize %
		\includegraphics[#1]{#2}
		\caption{#3}
		#4
	\end{figure}
}

\newcommand{\figMultCenterHere}[2]{
	\begin{figure}[H]
		\centering %
		\scriptsize %
		#1
		#2
	\end{figure}
}

% End of figures

\global\long\def\mymatrix#1{\boldsymbol{#1}}

\global\long\def\myvec#1{\boldsymbol{#1}}

\global\long\def\mapvec#1{\boldsymbol{#1}}

\global\long\def\dotproduct#1{\langle#1\rangle}

\global\long\def\norm#1{\left\Vert #1\right\Vert }

\global\long\def\sgn{\operatorname{sgn}}

\global\long\def\diag#1{\operatorname{diag}\left(#1\right)}

\global\long\def\diagterm#1{\mathfrak{D}\negmedspace\left\{  \negthinspace#1\negthinspace\right\}  }

\global\long\def\trace#1{\operatorname{tr}\left(#1\right)}
\begin{comment}
Trace of matrix \#1
\end{comment}

\global\long\def\inner#1#2{\langle#1,#2\rangle}
\begin{comment}
Operation: inner product (dot product)
\end{comment}

\global\long\def\outer#1#2{#1\times#2}
\begin{comment}
Operation: Outer product (cross product)
\end{comment}

\global\long\def\skew#1{{\left\{  #1\right\}  ^{\times}} }
\begin{comment}
Skew Symmetric Matrix
\end{comment}

\global\long\def\skewsymproduct#1{\ensuremath{\left\lfloor #1\right\rfloor _{\times}}}
\begin{comment}
Skew-symmetric matrix associated to cross-product with vector \#1
\end{comment}

\global\long\def\imi{\hat{\imath}}

\global\long\def\imj{\hat{\jmath}}

\global\long\def\imk{\hat{k}}

\global\long\def\imvec{\boldsymbol{\imath_{m}}}

\global\long\def\dq#1{\underline{\boldsymbol{#1}}}

\global\long\def\quat#1{\boldsymbol{#1}}

\global\long\def\dualvector#1{\underline{\boldsymbol{#1}}}

\global\long\def\dual{\varepsilon}

\global\long\def\mydual#1{\underline{#1}}

\global\long\def\hamilton#1#2{\overset{#1}{\operatorname{\mymatrix H}}\left(#2\right)}

\global\long\def\hamifour#1#2{\overset{#1}{\operatorname{\mymatrix H}}_{4}\left(#2\right)}

\global\long\def\hami#1{\overset{#1}{\operatorname{\mymatrix H}}}

\global\long\def\tplus{\dq{{\cal T}}}

\global\long\def\getp#1{\operatorname{\mathcal{P}}\left(#1\right)}

\global\long\def\getd#1{\operatorname{\mathcal{D}}\left(#1\right)}

\global\long\def\swap#1{\text{swap}\{#1\}}

\global\long\def\real#1{\operatorname{\mathrm{Re}}\left(#1\right)}

\global\long\def\imag#1{\operatorname{\mathrm{Im}}\left(#1\right)}

\global\long\def\vector{\operatorname{vec}}

\global\long\def\vectorinv{\underline{\operatorname{vec}}}

\global\long\def\mathpzc#1{\fontmathpzc{#1}}

\global\long\def\cost#1#2{\underset{\text{#2}}{\operatorname{\text{cost}}}\left(\ensuremath{#1}\right)}

\global\long\def\dualquaternion#1{\underline{\boldsymbol{#1}}}

\global\long\def\quaternion#1{\boldsymbol{#1}}

\global\long\def\sphereset{\mathcal{S}}

\global\long\def\closedballset{\ensuremath{\mathbb{B}}}
\begin{comment}
The closed unit ball in the Euclidean norm of convenient dimension
\end{comment}

\global\long\def\rationalset{\ensuremath{\mathbb{Q}}}
\begin{comment}
The set of rational numbers
\end{comment}

\global\long\def\integerset{\ensuremath{\mathbb{Z}}}
\begin{comment}
The set of integer numbers
\end{comment}

\global\long\def\naturalset{\ensuremath{\mathbb{N}}}
\begin{comment}
The set of natural numbers
\end{comment}

\global\long\def\realset{\ensuremath{\mathbb{R}}}
\begin{comment}
The set of real numbers
\end{comment}

\global\long\def\complexset{\ensuremath{\mathbb{C}}}
\begin{comment}
The set of complex numbers
\end{comment}

\global\long\def\dualset{\mathbb{D}}
 %
\begin{comment}
The set of dual numbers
\end{comment}
{} 

\global\long\def\quatset{\ensuremath{\mathbb{H}}}
 %
\begin{comment}
The set of quaternions
\end{comment}

\global\long\def\purequatset{\ensuremath{\mathbb{H}_{0}}}
 %
\begin{comment}
The set of pure quaternions
\end{comment}

\global\long\def\unitquatset{\mathcal{S}^{3}}
 %
\begin{comment}
The set of quaternions with unit norm, i.e. the unit 3-sphere
\end{comment}

\global\long\def\dualquatset{\mathbb{H}\otimes\mathbb{D}}
 %
\begin{comment}
The algebra of dual quaternions
\end{comment}
{} %
\begin{comment}
Selig's notation
\end{comment}

\global\long\def\puredualquatset{\ensuremath{\mathbb{D}_{V}}}
 %
\begin{comment}
The set of pure dual quaternions\textemdash dual vectors
\end{comment}

\global\long\def\unitdualquatset{\dq S}
\begin{comment}
The set of dual quaternions with unit norm, i.e. the Study set
\end{comment}
{} 

\global\long\def\StudySet{\mathcal{S}^{3}\otimes\mathbb{D}}
 %
\begin{comment}
The set of dual quaternions with unit norm
\end{comment}
{} %
\begin{comment}
Podemos usar também $\mathcal{S}_{3}\rtimes\mathbb{R}^{3}$
\end{comment}

\begin{comment}
\global\long\def\unitdualquatset{\mathcal{S}^{3}\otimes\mathbb{D}}
 
\end{comment}
\begin{comment}
The set of unit dual quaternions
\end{comment}

